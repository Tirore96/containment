\dawit{This section comes right after introducing new rules which we
claim allow more effecient coercions. Effeciency is related to interpretation}

The coercion system is sound if all derivable coercions are language containments:
\[\contains{}{c}{E}{F} \implies E \subseteq F\]
This is closely related to the interpretation of a coercion $\interp{\dslcom{c}} : \type{E} \rightarrow \type{F}$ due to a correspondance between language membership $s \in A$ and type inhabitance $\mathsf{t} : \type{A}$
\begin{lemma}\label{lem:types}
$\mathsf{t} : \type{A}$ implies $\flatten{\mathsf{t}} \in A$\\
$s \in A$ implies $\exists \mathsf{t}.~\mathsf{t} : \type{A}$ and $\flatten{\mathsf{t}} = s$
\end{lemma}
Lemma (\ref{lem:types}) allows us to prove soundness, by defining a terminating interpretation function from coercion derivations $\contains{}{c}{E}{F}$ to coercions in the function space $\type{E} \rightarrow \type{F}$.
% \begin{lemma}
% The existence of an interpretation function $\interp{ \contains{}{c}{E}{F}} : \{f : \type{E} \rightarrow \type{F}~ |~ \pres{f} \}$ implies soundness of the coercion system
% \end{lemma}
We now focus on defining an interpreter and get soundness for free. In Figure (\label{fig:interp}) the equtional rules of the interpreter are given. The last for equalities capture recursion, the first being Grabmeyer recursion with a termination argument known as the $0$-measure $|v|_0$ on parse trees. The last three equalities define the semantics of $\dslcom{peek}$ as a recursive function and it has a termination measure known as $1$-measure $|v|_1$ 
\begin{definition}[Measures on parse trees]
We define the two measures $|\cdot|_0$ and $|\cdot|_1$. For the cases where the two measures coinductive, we simply write $|\cdot|$.
  \begin{displaymath}
    \begin{array}{lll}
\onem{1} = 1 \qquad \zerom{1} = 0
\\\\
 \measure{0} = 0 \qquad \measure{a} = 1 \qquad \measure{inl~ t} = \measure{t} \qquad \measure{inr~ t} = \measure{t} \qquad  \measure{(v,w)}=\measure{v}+\measure{w} \qquad  \measure{fold~v}=\measure{v}\qquad

 %     \measure{(v,w)}=\measure{v}+\measure{w} \\\\ 

 %    \onem{0} = 0 \qquad \onem{1} = 1 \qquad \onem{a} = 1 \qquad \onem{inl~ t} = \onem{t} \qquad \onem{inr~ t} = \onem{t} \qquad
 %     \onem{(v,w)}=\onem{v}+\onem{w} \\\\
 % \onem{fold~v}=\onem{v}
    \end{array}
  \end{displaymath}
\end{definition}
We can verify that the course-of-values for $\dslcom{peek c}$ respects the $\zerom{\cdot}$ measure because in the recursive case \[\irule{(peek c)}{fold (inr (v,x))}{(peek c)(x)}\quad \text{if }\irule{c}{v}{inl ()}\]
We have that 
\[ \zerom{x} < \zerom{fold~ (inr~ v,x)}\]
We can also verify that Grabmeyer recursion satisfies the $\onem{\cdot}$ measure because none of the rules increases the $\onem{\cdot}$ measure, and because of the discharge rule:
\[\infer{\contains{\Gamma, f : A \leq B, \Gamma'}{id \times f}{a \times A}{a \times B}}{}\]
This means that if the application $\dslcom{fix f.c}(t)$ recurses, then reduction will lead to a term $\dslcom{f}(a,t')$ where $\onem{t'} < \onem{(a,t')}\leq \onem{t}$\\
\subsection{Interpretation}
We define the interpretation function $\llbracket \cdot \rrbracket_{\sigma_{\Gamma}}$ that translates a coercion derivation $\containsG{c}{A}{B}$ to a function on parse trees $\interp{\dslcom{c}}_{\sigma_{\Gamma}} : \mathcal{T}\llbracket A \rrbracket  \rightarrow \mathcal{T}\llbracket B \rrbracket$. The subscript, $\sigma_{\Gamma} : \{ \dslcom{f} ~|~\dslcom{f} : A \leq B  \in \Gamma \ \} \rightharpoonup  (\type{A} \rightarrow \type{B})$, is a partial map from assumptions $\dslcom{f} : A \leq B  \in \Gamma$, to functions on parse trees $\type{A} \rightarrow \type{B}$. 

The interpretation rules for all context rules as well as axioms, such as $\dslcom{shuffle}$, are exactly those presented in Henglein and Nielsen. As an example, a rule for shuffle is 
\[\interpsig{shuffle}{inl~v}{inl~(inl~v)} \]
To save space we omit interpretation rules for context rules and axioms, refering to Definition 16. in Henglein and Nielsen. Unlike Henglein and Nielsen who define interpretation of fix as $\interp{\fix ~ \dslcom{f.c}}(\mathsf{v})=\interp{\dslcom{c}[\fix~\dslcom{f.c} \backslash \dslcom{f}]}(\mathsf{v})$, we define the interpretation function with an environment $\interp{\cdot}_{\sigma_{\Gamma}}$ because this allows a precise definition that close to the Coq implementation.

In the fix rule, we write $\mathbf{f} \mapsto f_{\mathsf{v}} \cdot \sigma_{\Gamma}$ for the partial function extension, defined when $\mathbf{f} \notin \mathsf{Dom}(f)$. We label $f$ with subscript $\mathsf{v}$ for termination. This is seen in the rule where the application $\sigma_{\Gamma}~\mathsf{f}(\mathsf{v'})$ is allowed only when the string size has decreased since the declaration of the fixpoint.
\begin{definition}[Interpretation]
Let $\interp{\dslcom{c}}_{\sigma_{\Gamma}} : \type{A} \rightarrow \type{B}$ be the interpretation of $\containsG{c}{A}{B}$. We define it as:   
%For $\dslcom{f} \notin \mathsf{Dom}(\sigma_{\Gamma})$ and $f : \type{A} \rightarrow \type{B}$ we write the extension of $\sigma_{\Gamma}$ with $\dslcom{f}$ as $ \dslcom{f} \cdot \sigma_{\Gamma}$.
%We now give the definition of $\llbracket \containsG{c}{A}{B} \rrbracket : \mathcal{T}\llbracket A \rrbracket  \rightarrow \mathcal{T}\llbracket B \rrbracket$, assuming $\sigma_{\Gamma}$.\\
  \begin{displaymath}
    \begin{array}{l}
% \irule{shuffle}{inl v}{inl (inl v)}\\
% \irule{shuffle}{inr v}{inl (inr v)}\\
% \irule{shuffle}{(inr (inr v))}{inr v}\\
% \iruleinv{shuffle}{(inl (inl v))}{inl v}\\
% \iruleinv{shuffle}{(inl (inr v))}{inr (inl v)}\\
% \iruleinv{shuffle}{(inr v)}{inr (inr v)}\\
% \irule{retag}{(inl v)}{inr v}\\
% \irule{retag}{(inr v)}{inl v}\\
% \iruleinv{retag}{}{retag}\\
% \irule{untagL}{(inr v)}{v}\\
% \irule{untag}{(inl v)}{v}\\
% \irule{untag}{(inr v)}{v}\\
% \irule{tagL}{(v)}{inl v}\\ 
% \irule{assoc}{(v,(w, x))}{((v, w), x)}\\
% \iruleinv{assoc}{((v, w), x)}{(v,(w, x))}\\
% \irule{swap}{(v,())}{((), v)}\\
% \iruleinv{swap}{((), v)}{(v,())}\\
% \irule{proj}{((), w)}{w}\\
% \iruleinv{proj}{(w)}{((), w)}\\
% \irule{distL}{(v, inl w)}{inl (v, w)}\\
% \irule{distL}{(v, inr x)}{inr (v, x)}\\
% \iruleinv{distL}{(inl (v, w))}{(v, inl w)}\\
% \iruleinv{distL}{(inr (v, x))}{(v, inr x)}\\
% \irule{distR}{(inl v, w)}{inl (v, w)}\\
% \irule{distR}{(inr v, x)}{inr (v, x)}\\
% \iruleinv{distR}{(inl (v, w))}{(inl v, w)}\\
% \iruleinv{distR}{(inr (v, x))}{(inr v, x)}\\
% \irule{wrap}{(v)}{fold v}\\
% \dslcominv{wrap}(\textsf{(v)})=\dslcominv{fold}\\
% \irule{id}{v}{v}\\
% \iruleinv{id}{}{id}\\
% \irule{(c; d)}{(v)}{d(c(v))}\\
% \irule{(c + d)}{(inl v)}{inl (c(v))}\\
% \irule{(c + d)}{(inr w)}{inr (d(w))}\\
% \irule{(c × d)}{(v, w)}{(c(v), d(w))}\\
% \irule{shuffle}{inl v}{inl (inl v)}\\
% \irule{shuffle}{inr v}{inl (inr v)}\\
% \irule{shuffle}{(inr (inr v))}{inr v}\\
% \iruleinv{shuffle}{(inl (inl v))}{inl v}\\
% \iruleinv{shuffle}{(inl (inr v))}{inr (inl v)}\\
% \iruleinv{shuffle}{(inr v)}{inr (inr v)}\\
\infer{\interp{\fix~\dslcom{f}.\dslcom{c}}_{\sigma_{\Gamma}}(v)=\rec{f}{x}{\interp{\dslcom{c}}(x)_{\mathsf{var~f} \mapsto f \cdot \sigma_{\Gamma}}}{f}(v)}{} \qquad
\infer{\interp{\mathsf{f}}_{\sigma_{\Gamma}}(a,v)=(a,(\sigma_{\Gamma}~\mathsf{f})(v))}{f \in \mathbf{Dom}(\sigma_{\Gamma})}
\\\\
\infer{\interpsig{peek c}{fold (inl ())}{inl ()}}{} \qquad
\infer{\interpsig{peek c}{fold (inr (v,x))}{(\dslcom{peek c})(x)}}{\interpsig{c}{v}{inl ()}}
\\\\
\infer{\interpsig{peek c}{fold (inr (v,x))}{inr (w,x)}}{\interpsig{c}{v}{inr~w}}
    \end{array}
  \end{displaymath}
\end{definition}
\begin{lemma}[Termination]
For any coercion derivation $\contains{}{c}{A}{B}$ and $\mathsf{t} : \type{A}$,\\ 
$\interp{c}_\emptyset(\mathsf{t})$ is defined  %and preserves the string of $\mathsf{t}$
\end{lemma}
\begin{proof}[Sketch]
Show $\containsG{c}{A}{B}$ implies $\interp{c}_{\sigma_{\Gamma}}(\mathsf{t})$ is defined, assuming that forall $\mathsf{f} : C \leq D \in \Gamma$ and $\mathsf{v} : \type{C}$, where $|\mathsf{v}|_0 < |\mathsf{t}|_0$, that $(\sigma_{\Gamma}~\mathsf{f})(\mathsf{v})$ is defined.\\
Proof by induction on $\containsG{c}{A}{B}$. By the IH it suffices to show $f_{\mathsf{t}}(\mathsf{v})$ is defined, assuming $|\mathsf{v}|_0 < |\mathsf{t}|_0$. One can show $f_{\mathsf{t}}(\mathsf{v})$ is defined with $|\cdot|_0$ as termination measure.\\
The variable rule and non-recursive rule of $\dslcom{peek}$ are immediate. The latter two rules of $\dslcom{peek}$ can be shown terminating using $|\cdot|_1$ as termination measure.
\end{proof}
\begin{lemma}[Soundness of interpretation]
 For any coercion derivation $\contains{}{c}{A}{B}$\\ 
$\interp{c}_\emptyset :\type{A} \rightarrow \type{B}$ is a coercion
\end{lemma}
\begin{corollary}[Soundness]
$\contains{}{c}{A}{B}$ implies $A \subseteq B$
\end{corollary}
%  \caption{Interpretation semantics:  $\Gamma \vdash \dslcom{c}(v)=v'$: We present only the new rules for $\dslcom{peek}$ and changed rule for ($\fix f$). All other rules are identical to the presentation by Henglein and Nielsen and to see these definitions we refer to their work.}
\subsection{Representation}
In Coq we represent a derivation of $\containsG{c}{A}{B}$ as the term  \textsf{c : dsl Gamma A B}.
\mycomment{change drop to peek in code}
\mycomment{Outcommented coq code due to minted package errors}
% \begin{minted}{Coq}
% Inductive dsl (Gamma: seq (regex * regex)) : regex -> regex -> Type := 
% | shuffle A B C : dsl Gamma ((A _+_ B) _+_ C) (A _+_ (B _+_ C))
% | dfix A B : dsl ((A,B):: Gamma) A B -> dsl Gamma A B.
% | var a A B :   (A,B) \in Gamma -> dsl Gamma (Event a _;_ A) (Event a _;_  B) 
% ...
% \end{minted}
% One builds a variable using two regular expressions \textsf{A} and \textsf{B} and a proof that their pair is in the environment. 
% \begin{minted}{Coq}
% Lemma pf_in : (A,B) \in [(A,B)]. Proof. ... Qed.
% Definition var_example : dsl [(A,B)] (a * A) (a * B)  := var a A B pf_in
% \end{minted}
Applying \textsf{dfix} extends the environment. In the derivation below, we \textsf{dfix} extends the empty environment \textsf{nil} with \textsf{((1 + Star a),(Star a))} and \textsf{var a (1 + Star a) (Star a) pf\_in2} refers to this entry.
% \begin{minted}{Coq}
% Lemma pf_in2 : ((1 + Star a),(Star a)) \in [(1 + Star a) (Star a)].
% Definition example : dsl nil (1 + Star a) (Star a) := 
% dfix (ctrans (peek cid)
%      (ctrans (cplus cid (var a (1+(Star a)) (Star a) pf_in2)) wrap))
% \end{minted}

% The type of the interpretation function is seen below
% \begin{minted}{Coq}
% Definition post {A : eqType} (r0 r1 : @regex A) (T : pTree r0) := 
%   { T' : pTree r1 | pTree_0size T' <= pTree_0size  T /\ pflatten T = pflatten T' }. 


% Fixpoint interp l r0 r1 (p : dsl l r0 r1) (T : pTree r0) 
%          (Gamma_f : forall x y,  (x,y) \in l -> 
%                     forall (T0 : pTree x), pRel0 T0 T -> post y T0) {struct p}:
%          post r1 T. 
% \end{minted}
% \mycomment{change semantic rules to include environment}
% \textsf{Gammaf} is used to interpret the variables, and its type ensures that it can only be used on parse trees of smaller $0$-size..


%\newcommand{\tabline}[11]{#1 & \dentry{#2}{#7} & \dentry{#3}{#8} & \dentry{#4}{#9} & \dentry{#5}{#10} & \dentry{#6}{#11}}



%%% Local Variables: 
%%% mode: latex
%%% TeX-master: "main"
%%% End:
