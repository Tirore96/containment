% This is samplepaper.tex, a sample chapter demonstrating the
% LLNCS macro package for Springer Computer Science proceedings;
% Version 2.21 of 2022/01/12
%
\documentclass[runningheads]{llncs}
%
\usepackage[T1]{fontenc}
% T1 fonts will be used to generate the final print and online PDFs,
% so please use T1 fonts in your manuscript whenever possible.
% Other font encondings may result in incorrect characters.
%
\usepackage{graphicx}
\usepackage{hyperref}
\usepackage{proof}
\usepackage{minted}
\usepackage{xcolor}
\usepackage{stmaryrd}
\usepackage{numprint}
\usepackage{amsmath}
\usepackage{amsfonts}
\newcommand\mycomment[1]{\textcolor{red}{#1}}
%\newcommand\mycomment[1]{}

% \begin{document}
\input macros

% Used for displaying a sample figure. If possible, figure files should
% be included in EPS format.
%
% If you use the hyperref package, please uncomment the following two lines
% to display URLs in blue roman font according to Springer's eBook style:
%\usepackage{color}
%\renewcommand\UrlFont{\color{blue}\rmfamily}
%\urlstyle{rm}
%
\begin{document}
%
\title{Computationally interpreting regular expression containment proofs to efficient procedures } %TODO Please add
%
%\titlerunning{Abbreviated paper title}
% If the paper title is too long for the running head, you can set
% an abbreviated paper title here
%
\author{Fritz Henglein\inst{1}\orcidID{0000-0001-5190-2125} \and
Dawit Tirore\inst{2,3}\orcidID{0000-0002-1997-5161} \and
Nobuko Yoshida\inst{3}\orcidID{0000-0002-3925-8557}}
%
\authorrunning{F. Henglein et al.}
% First names are abbreviated in the running head.
% If there are more than two authors, 'et al.' is used.
%
\institute{University of Copenhagen, Universitetsparken 1, 2100 Copenhagen, Denmark
\email{henglein@diku.dk}
 \and 
IT University of Copenhagen, Rued Langgaards Vej 7, 2300 Copenhagen, Denmark
\email{dati@itu.dk}
 \and 
University of Oxford, Wellington Square, Oxford OX1 2JD, England
\email{nobuko.yoshida@cs.ox.ac.uk}
}
%Springer Heidelberg, Tiergartenstr. 17, 69121 Heidelberg, Germany
% \email{lncs@springer.com}\\
% \url{http://www.springer.com/gp/computer-science/lncs} \and
% ABC Institute, Rupert-Karls-University Heidelberg, Heidelberg, Germany\\
% \email{\{abc,lncs\}@uni-heidelberg.de}}
%
\maketitle    



%TODO mandatory: add short abstract of the document
\begin{abstract}
% Regular expressions are ubiqutious. They are theoretically relevant to automata theory for being syntactic representations of finite state automata and essential in practice for text processing. Theory and practice share an extensional view of regular expressions, caring more about the binary property of whether a string matches an expression, and less about the more informative property of how it matches the expression as a parse tree. Regular expressions can be viewed as types that have parse trees as their terms and Henglein and Nielsen demonstrate how these parse trees can represent compressed text. In most typed programming languages one expects type checking to work up to subtyping. Incorporating \textit{regular expressions as types} into such a language requires coercive subtyping to preserve that the only terms of regular expression types are their parse trees. Henglein and Nielsen show how such coercions can be derived by a computational interpretation of intuitionistic proofs of regular expression containments. They prove their system sound and complete with respect to language containment. We extend on their work by proving that efficient coercions can be synthesized efficiently and mechanize this result, along with soundness and completeness, in the Coq proof assistant. As part of the solution to synthesis, we also introduce a technique for mechanizing decidability of regular expression containment and equivalence in under 900 loc. Finally we extract the synthesis procedure and benchmark the performance of the coercions.
% \\\\
 It is known that proofs of membership of strings in (the language denoted by) a regular expression are in one-to-one correspondence to parse trees; and that a formal proof of a regular expression containment $A \leq B$ in an axiomatization of regular expression containment can be operationally interpreted as a coercion, a function that maps any proof of containment of a string in $A$ into a proof of containment of the same string in $B$.  Different containment proofs may yield extensionally different functions, however, since regular expressions, considered as grammars, can be ambiguous. Additionally, even if extensionally the same, one proof of containment may yield a slow implementation, another an efficient one.

In this paper we prove that efficient coercions can be synthesized efficiently and mechanize this result in Coq, along with soundness (only terminating coercions are inferred) and completeness (for each valid containment at least one coercion is derivable). We extract our efficient coercion synthesis procedure and benchmark the performance of the synthesized coercions. We also present a technique used in the synthesis procedure, that can prove regular expression equivalence in under 900 lines of code. This is to the best of our knowledge, the shortest mechanized proof in the literature for such a decision procedure.

\end{abstract}

 \section{Introduction}\label{sec:introduction}
\input introduction  

\section{Background}\label{sec:background}
\input background

\section{Faster Grabmeyer Coercions, A New Coercion Language}\label{sec:language}
\input language

\section{Deciding contaiment}\label{sec:decide}
\input decide

\section{Soundness and interpretation} \label{sec:soundness}
\input soundness

\section{Benchmarking}
\input bench

\section{discussion and related work}\label{sec:discussion}
\input discussion


\bibliography{ref}  

\end{document}
