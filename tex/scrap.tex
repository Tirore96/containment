%
% Regular expressions is inductive syntax that represents finite state automaton. Regular expressions equivalence and inequalities are well understood and studied. Often these proof systems are considered in a proof-irrelevant way, not caring what the inference tree is, only that it exists. Brandt and Henglein [cite] show how regular expression containment proofs can be interpreted to functional programs on parse trees. Such interpretation has applications in string compression, which we shall explain more later. The rules of their containment axiomatization corresponds to an intrinscially typed domain specific language. They derive the constructs of this language by combining the rules of idempotent semirings, unfolding $A`* \leq 1 +  A \times A^*$, with a powerful fix rule that is a of coinductive nature. 
%
% Our goal is effecient coercions on parse trees with applications for string compression. 
% Existing work by Brandt and Henglein formalize this using a powerful fix rule, whose soundness they ensure by instantiating it with a proper side condition. This side condition makes the coercion proofs less compositional and thus harder (as well as more expensive) to synthesize. This side condition checks for contractiveness. We derive a similar coercion system with a much weaker (but compositional) fixpoint rule that both simplifies (and makes more effecient) the synthesis of coercions, and implies contractiveness. The system is inspired by parameterized coinduction and allows us simple intepretation (our definition is good for simple termination arguments) and synthesis. These simple approaches would not have worked in Brandt and Henglein's origina system [detail why].\\\\
% We derive this inductive coercion system in the following steps: 
% \begin{enumerate}
% \item The coercion system is an axiomatization of language containment, which can be obtained by adding a few rules to axiomatizations of language equivalence. Starting with Grabmeyer's coinductive characterization, we prove it equivalent to another coinductive characterization that is free of operational notions. 
% \item We then prove this equivalent to an inductive version of this (why is it important to go inductive? Depends on the performance between the two dsls). 
% \item The inductive system for language equivalence is the altered to create one for language containment, and we go from Prop to Type.
% \end{enumerate}
% 
 





\section{Preliminaries}
\mycomment{double check it is 900 lines it takes to prove this}
Cardelli and Fresch introduce regular expressions as types.
\begin{definition}[Regular expression and semantics]
Regular expressions over a finite base set $a$ are given by the syntax:\\
$A ::= A + B~| A \times B ~|~A ^* ~|~ 1 ~|~ 0 ~|~ a$\\
Matching is defined as:\\
\begin{displaymath} 
\infer{\match{\epsilon}{1}}{}\qquad
\infer{\match{s s'}{A \times B}}{\match{s}{A} & \match{s'}{B}} \qquad
\infer{\match{s} {A + B}}{\match{s}{A}} \qquad
\infer{\match{s} {A + B}}{\match{s}{B}} \qquad
\infer{\match{\epsilon}{A^*}}{}\qquad
\infer{\match{s s'}{A^*}}{\match{s}{A} & \match{s'}{A^*}}\qquad
\end{displaymath}
Intrinsically typed parse trees:
\begin{displaymath}
\begin{array}{l}
\infer{\oft{()}{1}}{} \qquad 
\infer{\oft{a}{a}}{} \qquad \infer{\oft{\pair{t}{t'}}{A \times B}}{\oft{t}{A} & \oft{t'}{B}}
\qquad \infer{\oft{\inl {t}}{A + B}}{\oft{t}{A}} \qquad
\infer{\oft{\inr {t}}{A + B}}{\oft{t}{B}}  \qquad
\infer{\oft{\fold{t}{A^*}}}{\oft{t}{1 + A \times A^*}}
\end{array}
\end{displaymath}
\end{definition}
\begin{definition}[Equivalence] \noindent \\
Language equivalence: $\forall s, \match s\in A \iff s \in B$\\
Language containment: $\forall s, s \in A \implies s \in B$
\end{definition}
\begin{definition}[Derivative (standard/partial) and nullariness]
\begin{displaymath}
\begin{array}{l}
\mynu{a} = \false \qquad
\mynu{\epsilon} = \true \qquad
\mynu{0} = \false
\\\\
\mynu{A + B} = \mynu A  \lor \mynu B \qquad
\mynu{A \times B} = \mynu A  \land \mynu B
\end{array}
\end{displaymath}
\begin{displaymath}
\begin{array}{l}
\derive {a}{1} = \qquad
\derive{a}{0} = 0 \qquad
\derive{a}{b} = \epsilon if a = b\qquad
\derive{a}{b} = 0 if a \neq b\qquad
\derive {a}{(A + B)} = \derive{a}{A} +  \derive{a}{B}\qquad
\\\\
\derive {a}{(A \times B)} = \derive{a}{A} \times B +  \derive{a}{B} ~if \mynu{A} \qquad
\derive {a}{(A \times B)} = \derive{a}{A} \times B ~if not \nu{A} \qquad
\derive {a}{(A^*)} = \derive{a}{A} \times A^*
\end{array}
\end{displaymath}
\end{definition}j
\begin{definition}[Parameterized coinduction]
  fill out....
\end{definition}
% \section{Axiomatizations:  Prior ones and a new one}
% The first axiomatization of equivalence was given by Salomaa [cite]. We shall do as Brandt and Henglein, refering jointly to the laws of semiring, equality and the unfold rule $A^* = 1 + A \times A^*$ as \textit{weak equivalence}. Salomaa provec soundness and completeness of the system $F_1$ which consists of the rules of weak equivalence with the rules in Figure (\ref{fig:salomaa}).
% \begin{definition}[Laws of idempotent semi-ring]
% \label{definition:ring}
% \begin{displaymath}
% \begin{array}{lll}
% \myaxiom{A + B + C}{A + (B + C)} \qquad  
% \myaxiom{A + B}{B + A} \qquad 
% \myaxiom{A + 0}{A} \qquad
% \myaxiom{A + A}{A} \qquad
% \\\\
% \myaxiom{A \times B \times C}{A \times (B \times C)} \qquad
% \myaxiom{1 \times A}{A} \qquad
% \myaxiom{A \times 1}{A} \qquad 
% \myaxiom{0 \times A}{0} \qquad
% \\\\
% \myaxiom{A \times 0}{0} \qquad
% \myaxiom{A \times (B + C)} {A \times B + A \times C} \qquad
% \myaxiom{(A + B)\times C}{A \times C + B \times C}
% \end{array}
% \end{displaymath}
% \end{definition}

% \begin{definition}[Laws of equality]
% \label{definition:equality}
% \begin{displaymath}
% \begin{array}{lll}
% \infer{A = A}{}\qquad
% \infer{A = B}{B = A}\qquad
% \infer{A = C}{A = B & B = C}
% \end{array}
% \end{displaymath}
% \end{definition}
% \begin{figure}
% \caption{Rules of Salomaa}
% \label{fig:salomaa}
% \begin{displaymath}
% \infer[A_{11}]{\eqBodyE{A^*}{B^*}}
%   {\eqBodyE{A}{B}} \qquad
% \myaxiomN{(1 + A)^*}{A^*} \qquad \infer[\mynu{F}=\false]{\eqBodyE{E}{F^* \times G}}{\eqBodyE{E}{F \times E + G}}
% \end{displaymath}
% \end{figure}
% Salomaa proved that from the rules of weak equivalence with the unfold-rule and $A_{11}$, one can derive the decmosition:
% \begin{lemma} \label{lem:salomaa}
% \[A = \mynu{A} + \Sigma_{a \in \Sigma} \derive{a}{A}\]
% \end{lemma}
% We will return to this decomposition again later.




